\documentclass[12pt]{article}
%%%%%%%%%%%%%
%windows 正常 ok
%\usepackage{ctex} % 加载中文支持宏包
%macos ok
\usepackage{fontspec}
\usepackage{xeCJK}  % 如果你需要支持中文
%%%%%%%%%%%%%

\usepackage[x11names]{xcolor} % 启用更多颜色选项
\usepackage{amsmath}      % 数学公式
\usepackage{graphicx}     % 图片插入
\usepackage{listings}     % 用于插入代码
\usepackage{hyperref}     % 超链接支持
\usepackage{geometry}     % 设置页边距
\usepackage{fancyhdr}     % 页眉页脚设置
\usepackage{tocbibind}    % 目录、参考文献包含在目录中
\geometry{a4paper, left=1in, right=1in, top=1in, bottom=1in}

% 设置列表中Java代码的样式
\lstdefinestyle{java}{
    language=Java,
    backgroundcolor=\color{lightgray}, % 背景颜色
    basicstyle=\ttfamily\footnotesize, % 字体
    breaklines=true,                   % 自动换行
    captionpos=b,                      % 标题位置
    numbers=left,                      % 行号位置
    numberstyle=\tiny\color{gray},     % 行号样式
    keywordstyle=\color{blue},         % 关键词颜色
    stringstyle=\color{red},           % 字符串颜色
    commentstyle=\color{green},        % 注释颜色
    frame=single                       % 边框
}

% 设置页眉和页脚
\pagestyle{fancy}
\fancyhf{}
\fancyhead[L]{文章标题} % 左边页眉
\fancyhead[C]{作者姓名} % 中间页眉
\fancyhead[R]{日期} % 右边页眉
\fancyfoot[C]{\thepage} % 页脚中间显示页码

\title{论文标题}
\author{作者姓名}
\date{\today}

\begin{document}

\maketitle

% 生成目录
\tableofcontents
\newpage

\section{引言}
这里是文章的引言部分。

\section{Java代码示例}
以下是一个嵌入的Java代码示例:

\begin{lstlisting}[style=java, caption={HelloWorld.java}]
public class HelloWorld {
    public static void main(String[] args) {
        System.out.println("Hello, world!");
    }
}
\end{lstlisting}

\newpage
\begin{lstlisting}[style=java, caption={HelloWorld.java}]
public class HelloWorld {
  public static void main(String[] args) {
        //你好世界
        System.out.println("Hello, world!");
    }
}
\end{lstlisting}


\section{图片插入示例}
下面是一个插入图片的示例:

\begin{figure}[h]
    \centering
    \includegraphics[width=0.5\textwidth]{9.png} % 确保替换为你的图片路径
    \caption{示例图片}
    \label{fig:example}
\end{figure}

\section{结论}
本文展示了如何嵌入Java代码和图片,同时提供了目录、章节标题等结构。

\end{document}

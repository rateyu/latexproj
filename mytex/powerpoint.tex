\documentclass[UTF8]{beamer}
\usepackage{listings}  % 导入 listings 包,处理代码高亮
%\usepackage{xcolor}    % 设置颜色
\usepackage[x11names]{xcolor} % 启用更多颜色选项
\usepackage{graphicx}      % 插入图片
\usepackage{amsmath}       % 数学公式
\usepackage{fontspec}      % 使用系统字体
\usepackage{xeCJK}         % 中文支持
\usepackage{geometry}      % 设置页面大小和边距
\usepackage{hyperref}      % 超链接
\usepackage{fancyhdr}      % 页眉页脚设置
\usepackage{tikz}          % 绘图
%\usepackage{color}         % 颜色支持


% 设置页面大小(A4 或 A5)及方向(纵向或横向)
\geometry{a5paper, left=1in, right=1in, top=1in, bottom=1in}  % 设置 A5 纸张
% \geometry{a4paper, left=1in, right=1in, top=1in, bottom=1in}  % 如果需要 A4,可以取消上一行注释,注释掉 A5 相关设置
% \geometry{landscape}     % 如果需要横向(横版),取消注释这行

% 设置中文字体
\setCJKmainfont{WenQuanYi Zen Hei}     % 设置宋体为中文主字体
\setCJKsansfont{KaiTi}     % 设置黑体为无衬线字体
\setCJKmonofont{FangSong}   % 设置仿宋为等宽字体

% 设置英文字体
\setmainfont{Times New Roman}  % 设置英文字体为 Times New Roman

% 页面布局
\beamertemplatenavigationsymbolsempty  % 禁用默认的导航符号
\setbeamertemplate{footline}[frame number]  % 页脚显示页码

% 设置幻灯片标题样式
\setbeamertemplate{title page}{
    \begin{center}
        \vspace{2cm}
        \textbf{\Huge \inserttitle} \\[1em]
        \textit{\Large \insertauthor} \\[1em]
        \insertdate
    \end{center}
}

% 设置幻灯片主题
\usetheme{Madrid}  % 可以选择其他主题,如 CambridgeUS, AnnArbor, Warsaw, etc.

\title{幻灯片标题}
\author{作者姓名}
\date{\today}

\begin{document}

\frame{\titlepage}  % 插入标题页

\begin{frame}
    \frametitle{第一张幻灯片}
    这是你的第一张幻灯片内容。你可以在这里写一些文本。
    
    \begin{itemize}
        \item 第一项内容
        \item 第二项内容
        \item 第三项内容
    \end{itemize}
\end{frame}

\begin{frame}
    \frametitle{数学公式示例}
    在这里插入一些数学公式,例如:
    \[
        E = mc^2
    \]
\end{frame}

\begin{frame}
    \frametitle{嵌入图片}
    \begin{figure}[h]
        \centering
        \includegraphics[width=0.6\textwidth]{9.png} % 替换为实际图片路径
        \caption{示例图片}
    \end{figure}
\end{frame}

\begin{frame}
    \frametitle{代码示例}
    下面是一个简单的代码示例:
    \begin{verbatim}
public class HelloWorld {
public static void main(String[] args) {System.out.println("Hello, World!");
}
}
    \end{verbatim}
\end{frame}

\begin{frame}
    \frametitle{结束语1}
    感谢大家的聆听!
\end{frame}

\end{document}

\section{常用Kubernetes命令总结}
{\footnotesize % 脚注大小
集群管理命令:

kubectl get nodes:查看集群中的所有节点。

kubectl get pods:查看所有Pod的状态。

kubectl get services:查看所有Service。

kubectl get deployments:查看所有Deployment。

Pod管理命令:

kubectl create -f pod.yaml:通过YAML文件创建Pod。

kubectl describe pod <pod-name>:查看指定Pod的详细信息。

kubectl delete pod <pod-name>:删除指定Pod。

kubectl logs <pod-name>:查看Pod中的容器日志。

Deployment管理命令:

kubectl create -f deployment.yaml:创建Deployment。

kubectl get deployments:查看所有Deployment。

kubectl set image deployment/<deployment-name> <container-name>=<image>:更新Deployment的镜像。

kubectl scale deployment <deployment-name> --replicas=<number>:调整Deployment副本数。

kubectl rollout status deployment/<deployment-name>:查看Deployment的滚动更新状态。

kubectl rollout undo deployment/<deployment-name>:回滚Deployment。

Service管理命令:

kubectl expose pod <pod-name> --port=<port>:为Pod暴露Service。

kubectl expose deployment <deployment-name> --type=LoadBalancer --port=<port>:为Deployment暴露LoadBalancer类型的Service。

kubectl get svc:查看所有Service的状态。

调度命令:

kubectl get pod <pod-name> -o wide:查看Pod的详细信息,包括所在节点。

kubectl logs <pod-name> -c <container-name>:查看指定容器的日志。

集群调试命令:

kubectl describe pod <pod-name>:获取Pod的详细描述信息。

kubectl get events:查看集群的事件,帮助排查问题。

kubectl exec -it <pod-name> -- /bin/bash:进入Pod中的容器进行交互。
}

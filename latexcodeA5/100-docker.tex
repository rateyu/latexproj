\section{Docker核心概念总结}
{\footnotesize % 脚注大小
Docker概述:

Docker是一个开源的容器化平台,用于自动化应用程序的部署、扩展和管理。它通过将应用程序及其所有依赖项打包成一个容器,确保它们在任何环境中都能一致地运行。

容器化:Docker容器是轻量级的虚拟化技术,允许在一个宿主操作系统上运行多个隔离的应用。

Docker的组成:

Docker Engine:包括客户端和服务器,负责容器的构建和管理。

Docker镜像:包含应用运行所需的所有文件、库、依赖等,是容器的可执行包。

Docker容器:镜像的运行实例,是可以启动、停止、删除和管理的应用环境。

Docker Hub:Docker的公共镜像仓库,用户可以从中获取共享的镜像。

常用Docker命令:

docker run <image>:从指定镜像创建并启动一个新的容器。

docker ps:查看当前运行中的容器。

docker stop <container>:停止正在运行的容器。

docker rm <container>:删除停止的容器。

docker pull <image>:从Docker Hub或指定仓库拉取镜像。

docker build -t <image-name> <dockerfile>:使用Dockerfile构建镜像。

docker exec -it <container> /bin/bash:进入容器进行交互式操作。

docker logs <container>:查看容器的日志输出。

docker images:查看本地存储的所有镜像。

Docker容器生命周期:

创建:使用docker run或docker create命令从镜像创建容器。

启动:容器启动后,可以使用docker start命令来启动一个已创建的容器。

停止:使用docker stop停止正在运行的容器。

删除:使用docker rm删除已停止的容器。

Docker网络:

Bridge网络:默认网络模式,用于容器间的通信,容器通过虚拟网桥互联。

Host网络:容器与主机共享网络栈,直接使用主机的IP和端口。

None网络:不为容器分配网络,容器无法与外界通信。

Docker的存储:

Volumes:Docker容器中的数据存储。数据保存在宿主机上,且容器重启时数据不会丢失。

Bind Mounts:将宿主机的目录挂载到容器内,容器的修改会反映到宿主机上。

Tmpfs:将数据存储在内存中,当容器停止时数据会丢失。

Docker的安全性:用户权限:Docker容器以root用户身份运行,但可以通过USER指令指定非root用户运行。安全扫描:使用docker scan命令扫描镜像中的安全漏洞。隔离:通过容器的命名空间和cgroups技术来实现容器之间的隔离。
}

\documentclass[10pt]{ctexart}
\usepackage{graphicx}     % 图片插入
\usepackage{xcolor}
\usepackage{textcomp}
\usepackage{url}
\usepackage{listings}
% a4版本
%\usepackage[margin=1.2in]{geometry}
% 下面是a5版本
\usepackage{geometry}     % 设置页边距
% 设置 A5 纸张尺寸和 1 英寸的边距
\geometry{a5paper, left=0.5in, right=0.5in, top=0.5in, bottom=0.5in}


\definecolor{darkgray}{rgb}{.4,.4,.4}

\lstdefinelanguage{JavaScript}{
  keywords={typeof, new, true, false, catch, function, return, null, catch, switch, var, if, in, while, do, else, case, break},
  ndkeywords={class, export, boolean, throw, implements, import, this},
  sensitive=false,
  comment=[l]{//},
  morecomment=[s]{/*}{*/},
  morestring=[b]',
  morestring=[b]"
}

\lstset{
    aboveskip={.3\baselineskip},
    basicstyle=\scriptsize\ttfamily\linespread{4},
    breaklines=false,
    columns=flexible,
    commentstyle=\color[rgb]{0.127,0.427,0.514}\ttfamily\itshape,
    escapechar=@,
    extendedchars=true,
    frame=single,
    identifierstyle=\color{black},
    inputencoding=latin1,
    keywordstyle=\color[HTML]{228B22}\bfseries,
    language=JavaScript,
    ndkeywordstyle=\color[HTML]{228B22}\bfseries,
    numbers=left,
    numberstyle=\tiny,
    prebreak = \raisebox{0ex}[0ex][0ex]{\ensuremath{\hookleftarrow}},
    stringstyle=\color[rgb]{0.639,0.082,0.082}\ttfamily,
    upquote=true,
    showstringspaces=false,
}

\lstset{literate=%
   *{0}{{{\color{darkgray}0}}}1
    {1}{{{\color{darkgray}1}}}1
    {2}{{{\color{darkgray}2}}}1
    {3}{{{\color{darkgray}3}}}1
    {4}{{{\color{darkgray}4}}}1
    {5}{{{\color{darkgray}5}}}1
    {6}{{{\color{darkgray}6}}}1
    {7}{{{\color{darkgray}7}}}1
    {8}{{{\color{darkgray}8}}}1
    {9}{{{\color{darkgray}9}}}1
}

\begin{document}
\title{用  \texttt{listings} 制作类 minted 风格}

\maketitle

不少用户不会配置 minted 环境,我们也制作了视频教程教大家如何配置,可以到这里看看视频配置。
\url{https://www.bilibili.com/video/BV1sT4y1A7KR}


如果还是不会配置,今天给大家一个用
listings 宏包定制的类 minted 风格,这样就不用配置环境,也可以体验到 minted 样式了。



显示效果如下:
\begin{lstlisting}
var loginLayer=(function(){
    var div=document.createElement("div");
    div.innerHTML="windows1";
    div.style.display="none";
    document.body.appendChild(div);
    return div;
})();
/*document.getElementById('loginBtn').onclick=function(){
    loginLayer.style.display="block";
};*/
alert(loginLayer.style.display);

    public class HelloWorld {
        public static void main(String[] args) {
            System.out.println("Hello, World!");
        }
    }
\end{lstlisting}

\newpage
\begin{figure}
    \centering
    \includegraphics[width=0.5\textwidth]{9.png} % 确保替换为你的图片路径
    % \includegraphics[width=\textwidth, height=6cm]{example.jpg}  % 设置最大宽度为页面宽度,并限制高度
    %\includegraphics[width=\textwidth, height=12cm]{9.png}  % 设置最大宽度为页面宽度,并限制高度
    \caption{示例图片}
    \label{fig:example}
\end{figure}


\clearpage  % 所有浮动体(如图片)被处理并开始新的一页


% 插入第一张图片,上方,靠左对齐,去掉标题和空白
\begin{figure}[t!]
    \vspace{-50pt}  % 向上移动图片
    \hspace{-20pt}  % 向左移动图片
    \raggedright  % 使图片靠左
    \includegraphics[height=9cm]{9.png}  % 调整图片高度
    %\caption*{}  % 无标题显示
    \vspace{-70pt}  % 减少图片下方的空白
\end{figure}

% 插入第二张图片,下方,靠左对齐,去掉标题和空白
\begin{figure}[h!]
    \raggedright  % 使图片靠左
    \includegraphics[height=9cm]{9.png}  % 调整图片高度
    %\caption*{}  % 无标题显示
    \vspace{-10pt}  % 减少图片下方的空白
\end{figure}

\clearpage

% \newpage
核心定制代码如下:
\small
\begin{verbatim}
\lstset{
    aboveskip={.3\baselineskip},
    basicstyle=\scriptsize\ttfamily\linespread{4},
    breaklines=false,
    columns=flexible,
    commentstyle=\color[rgb]{0.127,0.427,0.514}\ttfamily\itshape,
    escapechar=@,
    extendedchars=true,
    frame=single,
    identifierstyle=\color{black},
    inputencoding=latin1,
    keywordstyle=\color[HTML]{228B22}\bfseries,
    language=JavaScript,
    ndkeywordstyle=\color[HTML]{228B22}\bfseries,
    numbers=left,
    numberstyle=\tiny,
    prebreak = \raisebox{0ex}[0ex][0ex]{\ensuremath{\hookleftarrow}},
    stringstyle=\color[rgb]{0.639,0.082,0.082}\ttfamily,
    upquote=true,
    showstringspaces=false,
}
\end{verbatim}

%\lstinputlisting{filename.js}
\end{document}
